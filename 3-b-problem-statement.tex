
\section{Problem Statement}

\label{whatsWrongWithIt}

% here's the outline; each block should be one sentence
As discussed in the previous sections, the work required to construct IAI systems we've seen isn't impractical.
Reducing or automating the work required to connect components would be beneficial - this is hindered by the specific domain in which we've seen IAI development carried out.
  Software components will be prepared with a diverse range of approaches, platforms, languages and environments.
    It should go without saying that the process of embedding or binding a component into the system shouldn't be a barrier when choosing components.
    To debug and correct problems it's essential to be able to decompose the system and remove components, without introducing "silent failures."
    % so - straight "write like this" isn't an option
    % this point feels a *bit* unimportant to the IAI community, but, seems like something that should be clarified for the PFP audience; i.e. we need to integrate the systems - can't "*just* write it correctly/in-Haskell"
  The development of such a system will involve continual updates to the subsystems that form components for the IAI.
    % can't treat systems as "buttoned-down" because ...
    Newer hardware may become available, machine learning models are tuned, bugs are fixed and component behaviours change throughout the project.
    Continuous refinement and adjustment are unavoidable, so, one can't always rely on things being "buttoned-down" once they've been observed functioning.
    Put more simply; constraints which aren't enforced by some manner of checks are likely to be violated.
    A function call that was expected to always pass whole numbers may begin using floating-point values as the timestamp it is based upon is made more accurate.
  % I'm "merging" the "large IO" and "hard to automate" blurbs
  The input and output pairs for a fully realised IAI is large and difficult to test automatically.
    Typically the input to such a system would be the audio and video recordings of a session.
    The outputs from an IAI session, at least in the case of a Virtual Human, would be (at the least) a similar audio-video stream.
    This interaction is formed by a user providing input to the system, which produces some output, which the user reacts to and produces new input to the system.
    Validating the outputs of the system (without decomposing it) would require some mechanism for comparing the outputs, which would likely be a human operator making regression testing prohibitive.
    Finally, the systems we've seen include "hidden inputs"\footnotemark in encapsulated modules which provide variations that aren't accounted for.
    Time is treated as an implicit input to the simulation and components, so, it's not explicitly accounted for in the system's behaviour.
\footnotetext {
  Pure Functional Programming might call these "side effects."
  In this case, we're considering system calls to ask for a random number or read the system clock.
}

    % - how about; can't *just* feed recorded video to the system since the feedback between the agent and the user would change and capturing that is essential
    % message queues don't "force" components to cycle; it's async
    % FRP means that the operations have to happen
    % need systems that REACT rather than do it in their own time at unpredictable rates
    % should I mention something about the fact we don't know when/how the messages will arrive?

In general, these types of issues are reminiscent of the sort of issues resolved with "Inversion of Control" and "Dependency Injection" approaches.\cite{fowler2004inversion}
  This states that a higher-level "policy" module shouldn't rely on a lower-level "mechanism" component; execution of logic shouldn't be tied to modifying specific files on disk.

Large systems typically encounter this class of problem, IAI systems face different challenges in resolving them.
    IAI systems use "interaction" with human operators as a component in the system.
        The system responds to the user, and the user responds to the system - this is a continual loop which is difficult to automate.
    The sensing systems typically used are imperfect.
        Lighting conditions can be different depending upon the user's posture or the time of day - making it difficult to analyse or even detect the user's face.
        Subtle variations in accents can make specific words consistently indecipherable - necessitating adjustments to the system or instructions to allow use to be carried out.
    The combined result of these factors leads to a system that might have the intended effect, or, it might not.
        We're advocating for a development method or architecture that allows continual adjustment and refinement while minimising potential uncertainty and risks.
        The theory is that the safeties of PFP will allow developers to rely on the computer validating their work as we make unplanned adjustments to correct unanticipated behaviours.
