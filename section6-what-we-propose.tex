

% re-state the points

\section{Proposed Approach}

Implementing a full Interactive Artificial Intelligence (IAI) system with Pure Functional Programming (PFP) would be one, admittedly unfeasible approach to this.
    Simply writing the system in a PFP method wouldn't prevent the issues identified from being present. 
        In effect, this approach would be trusting that the same mistakes won't be made again, and, that the development team will be able to track all details of their work correctly this time.
    While the concept of Turing Completeness\footnotemark would argue that it's possible to do what we describe in any language, we argue that with a PFP the task is simpler to finish as the implementation is checked by the language to an extent that it wouldn't be in an impure language.
    \footnotetext{
        "Turing-Completeness" here refers to the notion that any program expressed in a "Turing-Complete" language can be mechanically translated or emulated in any other Turing-Complete language.
    }
    Functional Purity allows us to reuse logic without the need to consider unseen side effects, and our goal is to use an approach that is, as described in\cite{perez2017testing} "functional purity at the system level."

In 2002 Gratch and others identified the need to develop reusable tools and modular architectures for the development of IAI.\cite{gratch_creating_2002}
They also identified that consistency and timing between separate modules as an issue that needs to be managed.
    While we cannot "enforce" handling of either of these, our approach should assist in handling concerns involved.
        We can't force a component to attach the same timestamps that we do, but, we can provide a sort fo "marshalling area" where the data from the component (including timestamps) is processed and passed into the system.
    By explicit handling how the data is "marshalled" before being passed into the simulation, or along to other bindings, we hope to make the process more consistent.

Functional Reactive Programming (FRP) offers a way to declaratively specify a system which behaves predictably.\cite{perez2017testing}
    The approach of FRP offers an aspect of referential transparency\footnotemark but at a system level rather than just at the level of individual statements.
    \footnotetext{
        When a piece of logic is "Referentialy Transparent" it will produce the same outputs when given the same inputs.
    }
    Functional reactive programming is mostly done by composing "signal functions" (SF) which consume some *event* and emit some *signal*.\footnotemark
    \footnotetext{I'm adopting the nomenclature mapping *event* and *signal* to the input and output from Signal Functions - others may use the terms in different ways.}
    An agent written in FRP would "open" several "Foreign Signal Functions" to communicate with external components bound to the system.
    The program would combine these, along with general "utility" constructs into a sort of "outer" SF representing the initial state of the agent.
    The program can then be started, collecting the specifies events (or non-events) from bindings, passing them into the "outer SF" and then forwarding any signals back out to the active components.
To perform this work, we're developing an implementation of a sort of "shell" responsible for generating PFP code to interface between the agent and any components.
    For deployment purposes; some method of "freezing" these interfaces will likely be used - the motivation for the code generation is to automate the step of "updating" something like this when a binding changes.

% describe the isolation of bindings

\subsection{Binding Components}

It is assumed that components won't be written in FRP, and will need some amount of adaptation to work with this system.
Each component should have some amount of "glue" code to expose available functionality to the system - which we refer to as the "binding."

We've found that message types can be extracted from compiled Scala classes using the JVM's run-time reflection.
    With this approach, the Scala definitions become the "master description" of how data will be passed in and out by the binding.
        Specifying these structures in the bindings, then generating the "headers" for the agent seemed like a logical method to carry out this step.
        If relying on reflection is undesirable or otherwise inconvenient; one could use some manner of Interface Definition Language to generate "both sides" of the data structures from one master description.
To pass data between components, and finally prepare the output - we used Scala's (for-yield, a) variation of "do-notation" to create a monad that can be invoked as needed to check for an event or receive a signal.
    \footnotetext{
        A very informal explanation of the Haskell do-notation syntax can be found here \url{http://learnyouahaskell.com/a-fistful-of-monads}.
    }
    Interestingly this allowed us to express parameters and dependencies between the components concisely.
    Listing \ref{lst:depa} illustrates a contrived example of this mechanism.
Each binding will be a singleton instantiated during shell creation.
    Each binding will be analysed (via reflection) to generate a functional "header" used to perform the actual invocations of functions on the instance.
    Any functionality desired by the agent is accessed through a "Foreign Signal Function" which acts as a sort of "handle" to the aforementioned monad.
    When the shell wishes to start a new loop or "cycle" is will "solve" all of these Monads to create and plugin whatever data they need to work and input they wish to send into the agent's simulation.

\begin{lstlisting}[caption={Scala for-yield to interact with the agent}, label={lst:depa},basicstyle=\small,language=Scala]

def getTime = System.currentTimeMillis()

// ////
// illustration of sending a timestamp event

  // record the start time lazily
  lazy val start = getTime

  // this "for loop" creates a sort of monadic query
  for {
    // when executed, this instance is created
    // ... and returned for simplicity
    time <- share(new TimeStamp(getTime - start))
    
    // previous line creates and shares a timeStamp

  } yield {
    // this line just returns the timeStamp event
    time
  }

// ////
// illustration of an event that needs the timestamp
  for {
    // this will wait for a timeStamp value
    time <- request[TimeStamp]
  } yield {
    
    // this event then returns an instance
    new SomeOtherData(comeComputation(time.age))
  }

// ////
// a signal handler that needs the timestamp
  for {
    // this will request the timestamp
    // ... after the agent has executed
    time <- request[TimeStamp]
  } yield {
    
    // the lambda consumes an instance of SignalValue
    (some: SignalValue) =>
      ??? // perform some handling of the signal
  }

\end{lstlisting}



\subsection{Agent Scenarios}

The "agents" themselves are pure functional programs written in a traditional pure functional programming language.
    We're currently using Idris, but, working to switch to PureScript as its tool set looks more suitable in practice.
Agents will be written against "headers" corresponding to bindings, these headers will define how the agent interacts with the IO from the shell and components.
Once written, the agent is compiled to a \lstinline{.js} program that can be loaded into the Java11/GraalJS interpreter.
    There's no reason that another compatible interpreter or platforms couldn't be used; this was chosen for simplicity 
When operating, the agent cannot open (or close) additional connections to components; this is intentional as it means missing resources can be detected on startup rather than during execution.


\subsection{System Lifecycle}


%    https://mermaid-js.github.io/mermaid-live-editor/#/edit/eyJjb2RlIjoiZ3JhcGggVERcblxuICAgIGdlbltnZW5lcmF0ZSBoZWFkZXJzXVxuICAgIG1ha2VbY29tcGlsZSBhZ2VudCB0byBgLmpzYF1cbiAgICBsb2FkW2xvYWQgYWdlbnQgaW50byBWTV1cbiAgICBleGVjdXRlW3J1biBgLmpzYCB0byBvcGVuIEZTRiBhbmQgZ2V0IGluaXRpYWwgYWdlbnQgc3RhdGVdXG4gICAgZXZlbnRbY29sbGVjdCBldmVudHMgZnJvbSBjb21wb25lbnRzXVxuICAgIHJlYWN0W2NvbnN1bWUgZXZlbnRzIGFuZCBjb21wdXRlIHNpZ25hbHNdXG4gICAgc2lnbmFsW2ZvcndhcmQgc2lnbmFscyBiYWNrIHRvIGNvbXBvbmVudHNdXG5cbiAgICBzdWJncmFwaCBidWlsZFxuICAgICAgICBnZW4tLT5tYWtlXG4gICAgZW5kXG5cbiAgICBzdWJncmFwaCBleGVjdXRpb25cbiAgICAgICAgZXZlbnQtLT4gcmVhY3RcbiAgICAgICAgcmVhY3QgLS0-IHNpZ25hbFxuICAgICAgICBzaWduYWwgLS0-IGV2ZW50XG4gICAgZW5kXG5cbiAgICBtYWtlIC0tPiBsb2FkXG5cbiAgICBzdWJncmFwaCBsYXVuY2hcbiAgICAgICAgbG9hZCAtLT4gZXhlY3V0ZVxuICAgIGVuZFxuXG4gICAgZXhlY3V0ZSAtLT4gZXZlbnRcbiAgICBcbiAgICBcbiAgICBcblxuXHRcdFx0XHRcdCIsIm1lcm1haWQiOnsidGhlbWUiOiJkZWZhdWx0In0sInVwZGF0ZUVkaXRvciI6ZmFsc2V9


\begin{figure}
  \caption{Proposed Lifecycle}
  \includegraphics[keepaspectratio, width=8cm]{content/execution}
  \label{fig:execution}
\end{figure}


Figure \ref{fig:execution} illustrates the proposed lifecycle of the system described here - omitting details that should be obvious to the reader.
    \footnote{
        Such as the instantiation of bindings, their threads and network connections, the creation of an ECMAScript engine, etc.
    }
    The startup creates and updates any headers to ensure that the agent is operating against an up to date version of the bindings.
        If the available components have changed (since the agent was written) - this will produce an immediate error on startup, rather than at a later date.
    The compiled \lstinline{.js} is executed which connects to all components and composes the initial state of the system.
    Fro that point on, the system enters an (intentionally) infinite loop where it collects events from components, "executes" the agent to consume those events, and forwards signals from the cycle back out to components as needed.

Data is passed in and out of the agent in strict cycles.
    This allows a measure of safety and checking.
    Each Foreign Signal Function (FSF) is expected to be invoked once and only once during each cycle.
    The agent's outputs are considered "stateless" so \textit{could} be executed multiple times with only the last value being relevant - however this \textit{feels} wrong.
    The agent's inputs are dependent on what was passed in from the components, so the meaning of "missing" a value or reading it multiple times is unclear; should the value be re-read, queued.
    We take a strict stance that the simulation, just like the bindings, must behave reliably when running.

On launch, the agent is required to "open" each input that it wishes to read from and each output it will provide.
    This is enforced by binding the "creation" function to the IO Monad and removing the foreign calls from the ECMAScript namespace once this step has been completed.
    If the compiled agent program is reliant upon something like a stale header; this will cause an immediate error since the corresponding ECMAScript functionality won't be defined.

The design isn't firmly tied to any of the tools used to build it.
    Easily embeddable ECMAScript engines\footnote{The library DukTape is designed for this type of use. \url{https://duktape.org/}} combined with tools to manipulate\footnote{The tool "Babel.JS" can "downgrade" newer JavaScript programs for older interpreters. \url{https://babeljs.io/}} the compiled \lstinline{.js} would allow the approach to be used on other platforms.
        % Some convention would need to be established to replace the event collection, but, that hardly seems like a barrier.
    The goal of this approach is to automate checking that ensues the framework, the AI, and the components are all operating in a manner that's easy to coordinate.

% [needs can be detected - limit IO between agent and shell to just these cycles]
%     [opening an io is "only allowed at the start" as it has a "side effect"]
%     [smaller input pairs for the scenarios]
%         [actually - a driver's IO can be recorded and replayed on its own]
%     [nothing to prevent re-running events and testing scenarios]
%     [nothing to prevent re-running events and testing component]
