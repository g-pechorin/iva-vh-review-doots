

% re-state the points

\section{Proposed Approach}

Implementing a full Interactive Artificial Intelligence (IAI) system with Pure Functional Programming (PFP) would be one, admittedly unfeasible approach to this.
    Simply writing the system in a PFP method wouldn't prevent the issues identified from being present. 
        In effect, this approach would be trusting that the same mistakes won't be made again, and, that the development team will be able to track all details of their work correctly this time.
    While the concept of Turing Completeness\footnotemark would argue that it's possible to do what we describe in any language, we argue that with a PFP the task is simpler to finish as the implementation is checked by the language to an extent that it wouldn't be in an impure language.
    \footnotetext{
        "Turing-Completeness" here refers to the notion that any program expressed in a "Turing-Complete" language can be mechanically translated or emulated in any other Turing-Complete language.
    }
    Functional Purity allows us to reuse logic without the need to consider unseen side effects, and our goal is to use an approach that is, as described in\cite{perez2017testing} "functional purity at the system level."

In 2002 Gratch and others identified the need to develop reusable tools and modular architectures for the development of IAI.\cite{gratch_creating_2002}
They also identified that consistency and timing between separate modules as an issue that needs to be managed.
    While we cannot "enforce" handling of either of these, our approach should assist in handling concerns involved.
        We can't force a component to attach the same timestamps that we do, but, we can provide a sort fo "marshalling area" where the data from the component (including timestamps) is processed and passed into the system.
    By explicit handling how the data is "marshalled" before being passed into the simulation, or along to other bindings, we hope to make the process more consistent.

Functional Reactive Programming (FRP) offers a way to declaratively specify a system which behaves predictably.\cite{perez2017testing}
    The approach of FRP offers an aspect of referential transparency\footnotemark but at a system level rather than just at the level of individual statements.
    \footnotetext{
        When a piece of logic is "Referentialy Transparent" it will produce the same outputs when given the same inputs.
    }
    Functional reactive programming is mostly done by composing "signal functions" (SF) which consume some *event* and emit some *signal*.\footnotemark
    \footnotetext{I'm adopting the nomenclature mapping *event* and *signal* to the input and output from Signal Functions - others may use the terms in different ways.}
    An agent written in FRP would "open" several "Foreign Signal Functions" to communicate with external components bound to the system.
    The program would combine these, along with general "utility" constructs into a sort of "outer" SF representing the initial state of the agent.
    The program can then be started, collecting the specifies events (or non-events) from bindings, passing them into the "outer SF" and then forwarding any signals back out to the active components.
To perform this work, we're developing an implementation of a sort of "shell" responsible for generating PFP code to interface between the agent and any components.
    For deployment purposes; some method of "freezing" these interfaces will likely be used - the motivation for the code generation is to automate the step of "updating" something like this when a binding changes.

% describe the isolation of bindings

\subsection{Binding Components}

It is assumed that components won't be written in FRP, and will need some amount of adaptation to work with this system.
Each component should have some amount of "glue" code to expose available functionality to the system - which we refer to as the "binding."

We've found that message types can be extracted from compiled Scala classes using the JVM's runtime reflection.
    With this approach, the Scala defintions become the "master description" of how data will be passed in and out by the binding.
        Specifying these structures in the bindings, then generating the "headers" for the agent seemed like logical method to carry out this step.
        If relying on reflection is undesireable or otherwise incvonvient; one could use some manner of Interface Defintion Language to generate "both sides" od the data structures from one master description.
To pass data between components, and finally prepare the output - we used Scala's (for-yield, a) varation of "do-notation" to create a monad that can be invoked as needed to check for an event or receive a signal.
    \footnotetext{
        A very informal explanation of the Haskell do-notation syntax can be found here \url{http://learnyouahaskell.com/a-fistful-of-monads}.
    }
    Interestingly this allowed us to express parameters and dependencies between the components in a concise way.
    Listing \ref{lst:depa} illustrates a contrived example of this mechanism.
Each binding will be a singleton instantiated durring shell creation.
    Each binding will be analysed (via reflection) to generate a functional "header" used to perform the actual invocations of functions on the instance.
    Any functionality desired by the agent is accessed through a "Foriegn Signal Function" which acts as a sort of "handle" to the aforementioned monad.
    When the shell wishes to start a new loop or "cycle" is will "solve" all of these Monads to create and plugin whatever data they need to work and input they wish to send into the agent's simulation.

\begin{lstlisting}[caption={Scala for-yield to interact with the agent}, label={lst:depa},basicstyle=\small,language=Scala]

def getTime = System.currentTimeMillis()

// ////
// illustration of sending a timestamp event

  // record the start time lazily
  lazy val start = getTime

  // this "for loop" creates a sort of monadic query
  for {
    // when executed, this instance is created
    // ... and returned for simplicity
    time <- share(new TimeStamp(getTime - start))
    
    // previous line creates and shares a timeStamp

  } yield {
    // this line just returns the timeStamp event
    time
  }

// ////
// illustration of an event that needs the timestamp
  for {
    // this will wait for a timeStamp value
    time <- request[TimeStamp]
  } yield {
    
    // this event then returns an instance
    new SomeOtherData(comeComputation(time.age))
  }

// ////
// a signal handler that needs the timestamp
  for {
    // this will request the timestamp
    // ... after the agent has executed
    time <- request[TimeStamp]
  } yield {
    
    // the lambda consumes an instance of SignalValue
    (some: SignalValue) =>
      ??? // perform some handling of the signal
  }

\end{lstlisting}



\subsection{Agent Scenario}

[only the agent is written in PFP]
    [Pareto Principle; 20\% of the population will control 80\% of the wealth]
        % I'm taking it as a given that the agents will have the bugs
    [???]
    [hopefully, the parts we're used to changing are the culprits]

- [The agents themselves will be written in perfectly normal Idris\footnote{... or PureScript, I'm trying to migrate the agent language for technical reasons.} and compiled to a `.js` blob which can be loaded into the Java11/GraalVM's interpreter.]
    - [The use of the two "Java" platforms tends to mislead people - the goal isn't to develop a new way to run JavaScript, and the Java Virtual Machine was chosen due to the build tools making testing simple.]


\subsection{describe execution}

[pass data back/forth in strict cycles]
    [mostly because we can, but, also makes life simpler]
    [strict rules are placed on how the bindings can act and react]
        [one in one out]
    [this should limit the amount of "stuff" that can go wrong ... right]
        [drivers can still ignore errors - but now the error is in one known place]
    [each driver components can/is isolated; making their lives easier]
[needs can be detected - limit IO between agent and shell to just these cycles]
    [opening an io is "only allowed at the start" as it has a "side effect"]
    [smaller input pairs for the scenarios]
        [actually - a driver's IO can be recorded and replayed on its own]
    [nothing to prevent re-running events and testing scenarios]
    [nothing to prevent re-running events and testing component]

    - [On launch, the agent must "open" each input that it wishes to read from and each output it will provide.]
        - [If the available components have changed (since the agent was written) - this will produce an immediate error on startup, rather than at a later date.]



\subsection{detail why this solves the problem}

- [The approach here isn't focussed on building "master APIs" for the lines of communication between components, but, rather a system to automatically generate and check the interconnections between those components.]
    - [The intent is to support flexible working; incompatible interface changes should cause type errors, and aberrant states can be reproduced and analysed.]
- [Agents Scenarios are not reliant upon an external encoding or communications system, nor can they easily buildup state that'll introduce aberrant behaviour - they stay "fresh" in a "just turned on" state.]

- [The shell we propose enforces consistent input and output operations from the agent.]
    - [If the agent fails to specify or read a value for something there will be an immediate error; enforcing consistent behaviour.]
- [These aspects are currently "enforced" by our implementation, prior work has identified a method to enforce it as a compile-time constraint.\cite{winograd2012wormholes}]

- [Additionally the strict delineations between "agent" and "shell" afford additional options for testing and diagnosis.]
    - [A "built-in" strict separation isn't required for "good" testing, but, it's helpful and considered a good practice anyway.]
- [The proposed component integration decouples them from the simulation - making "headless running" of the bound components "practical" compared to a more "tight" integration.]
    - [Nothing stops component developers from doing this on their own, but, this enforces a consistent structure so that one might be able to test them similarly.]
- [Finally, as noted elsewhere, abstracting the FFI into and out of the agent as streams of data means that the agent (or components) can be tested against previous sessions, and, re-played to re-reach a previous state.]
    - [Again; due to Turning Completeness - any system could do this with enough work, but, this approach "supports" it.]

- [Prior work has shown both that FRP approaches are quite suited to testing\cite{perez2017testing} and that automated testing of PFP itself can be quite useful in determining "edge cases"\cite{claessen2011quickcheck} which would otherwise be a challenge to diagnose.]
    - [The ability to abstract and encode all invocations coming in and out of the system as distinct data points are key to this.]