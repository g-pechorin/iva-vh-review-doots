
\section{Conclusion and Further Work}

In this paper, we have shown a concise approach to constructing a modular framework for building Virtual Humans or Interactive AI.
The approach here isn't focused on building "master APIs" or generic utility components for the construction of these systems, but, rather a system to automatically generate and check the interconnections between those components.
  The intent is to support flexible working; incompatible interface changes should cause type errors, and aberrant states can be reproduced and analysed.
  The use of functional programming techniques allows us to make assumptions about the stability of the constructs without needing to manually audit modules for inconsistencies.
Agent's Scenarios are not reliant upon an external encoding or communications system, nor can they easily buildup state that will introduce aberrant behaviour - they stay "fresh" in a "just turned on" state.

The shell we propose enforces consistent input and output operations from the agent.
  If the agent fails to specify or read a value for something there will be an immediate error; enforcing consistent behaviour.
These aspects are currently "enforced" by our implementation, prior work has identified a method to enforce it as a compile-time constraint.\cite{winograd2012wormholes}

Additionally the strict delineations between "agent" and "shell" afford additional options for testing and diagnosis.
  A "built-in" strict separation isn't required for "good" testing, but, it's helpful and considered a good practice anyway.
The proposed component integration decouples them from the simulation - making "headless running" of the bound components "practical" compared to a more "tight" integration.
  Nothing stops component developers from doing this on their own, but, this enforces a consistent structure so that one might be able to test them similarly.
Finally, as noted elsewhere, abstracting the FFI into and out of the agent as streams of data means that the agent (or components) can be tested against previous sessions, and, re-played to re-reach a previous state.
  Again; due to Turning Completeness - any system could do this with enough work, but, this approach "supports" it.

Prior work has shown both that FRP approaches are quite suited to testing\cite{perez2017testing} and that automated testing of PFP itself can be quite useful in determining "edge cases"\cite{claessen2011quickcheck} which would otherwise be a challenge to diagnose.
  The ability to abstract and encode all invocations coming in and out of the system as distinct data points are key to this.

The concepts in this paper have been demonstrated in isolation, and work is ongoing to integrate and validate them.
  Certain limitations need to be addressed; the most glaring is that the implementation suffers from an inevitable stack overflow error as the chosen ECMAScript engine cannot perform tail call elimination.
    The way that the loops are implemented causes each iteration to push new stack-frame for the next cycle, from which the program will never return.
    This is a well-understood problem in the functional programming community, but, was discovered late in the project.
    The intended solution is a technique called "trampolining" which is being implemented for the next iteration of the system.
  To assess the ease of use for this approach, we'd like to recruit participants to make use of the system in an informal setting.
    This will likely involve liaising with researchers in our institution and providing them with a demonstration of the system's capabilities before tasking them with producing some artefact that builds on that.
    Following such a session - we'd like to conduct an exit interview and see if aspects should be highlighted or dismissed.
  Finally, the ability to record and replay or even resume sessions continues to capture the attention of peers whom we've discussed the system with.
    Providing tools to simplify this task, or even automate performing and analysing the results of it would be desirable.
